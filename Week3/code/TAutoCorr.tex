\documentclass[12pt]{article}
\title{Report of "Autocorrelation in weather"}
\author{Ran Tao}
\date{October 2018}
\begin{document}
  \maketitle

  \begin{abstract}
    This paper report the results of the "Autocorrelation in weather" project and give interpretation
  \end{abstract}

  \section{Introduction}
    Autocorrelation in weather: The goal is to answer the question: Are temperatures of one year significantly correlated with the next year (successive years), across years in a given location? For this, you need to calculate the correlation between n−1 pairs of years, where n is the total number of years. However, you can't use the standard p-value calculated for a correlation coefficient, because measurements of climatic variables in successive time-points in a time series (successive seconds, minutes, hours, months, years, etc.) are not independent. 

  \section{Methods}
    1.Compute the appropriate correlation coefficient between successive years
    2.Repeat this calculation 10000 times by -- randomly permuting the time series, and then recalculating the correlation coefficient for each randomly permuted year sequence
    3.Then calculate what fraction of the correlation coefficients from the previous step were greater than that from step 1 (this is your approximate p-value)

  \section{results}
    Pvalue is not a constant value because the samples are random. Pvalue = n*e-04(n is not constant but less than 10)

  \section{interpretation}
    P value is so small that is less than 0.001, so the possibility of the correlation coefficient between successive year larger than between random year is higher than 99.9 of one hundred percent. Therefore the temperatures of one year significantly correlated with the next year (successive years) 
  
\end{document}
\grid